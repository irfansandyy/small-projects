\documentclass[12pt,a4paper]{report}
\usepackage[utf8]{inputenc}
\usepackage{graphicx}
\usepackage{amsmath}
\usepackage{geometry}
\usepackage{hyperref}
\usepackage{caption}
\usepackage{subcaption}
\usepackage{lipsum} % untuk dummy text

\geometry{margin=3cm}

\title{Laporan Proyek Akhir}
\author{Mochammad Irfan Sandy}
\date{\today}

\begin{document}

\maketitle
\tableofcontents
\listoffigures
\listoftables
\newpage

\chapter{Pendahuluan}

\section{Latar Belakang}
\lipsum[1]

\section{Rumusan Masalah}
\begin{itemize}
    \item Bagaimana cara merancang sistem distribusi pesan menggunakan load balancer?
    \item Bagaimana efektivitas metode round-robin dalam pembagian beban kerja?
\end{itemize}

\section{Tujuan Penelitian}
\lipsum[2]

\chapter{Tinjauan Pustaka}

\section{Load Balancer}
Load balancer adalah komponen penting dalam sistem terdistribusi. Menurut, load balancing bertujuan membagi beban secara merata.

\section{Inter Process Communication (IPC)}
Beberapa metode IPC:
\begin{enumerate}
    \item Message Queue
    \item Shared Memory
    \item Semaphore
\end{enumerate}

\chapter{Metodologi}

\section{Arsitektur Sistem}
Berikut adalah arsitektur sistem yang diusulkan.

\begin{figure}[h!]
    \centering
    \caption{Diagram arsitektur sistem}
    \label{fig:arsitektur}
\end{figure}

\section{Desain Algoritma}
Algoritma menggunakan metode round-robin:

\begin{verbatim}
current_worker = (current_worker + 1) % total_workers;
send_message(current_worker, message);
\end{verbatim}

\chapter{Implementasi dan Hasil}

\section{Lingkungan Implementasi}
\begin{itemize}
    \item Bahasa: C
    \item OS: Linux Ubuntu 22.04
    \item Komunikasi: IPC Message Queue
\end{itemize}

\section{Contoh Tabel Log}
\begin{table}[h!]
\centering
\begin{tabular}{|c|c|c|}
\hline
No & Waktu & Worker Tujuan \\ \hline
1 & 10:00 & Worker 1 \\
2 & 10:01 & Worker 2 \\
3 & 10:02 & Worker 3 \\
\hline
\end{tabular}
\caption{Log distribusi pesan}
\label{tab:log}
\end{table}

\chapter{Kesimpulan dan Saran}

\section{Kesimpulan}
\lipsum[3]

\section{Saran}
\lipsum[4]

\addcontentsline{toc}{chapter}{Daftar Pustaka}
\begin{thebibliography}{9}
\bibitem{tanenbaum2016}
Andrew S. Tanenbaum and Maarten van Steen, \textit{Distributed Systems}, 3rd ed., Pearson, 2016.

\bibitem{linuxipc}
Michael Kerrisk, \textit{The Linux Programming Interface}, No Starch Press, 2010.
\end{thebibliography}

\end{document}
